% package pour les mathématiques
\usepackage{amsmath,amssymb,amsfonts,amsthm}
\usepackage{mathrsfs}
\usepackage{stmaryrd} % pour les intervalles d'entiers
\usepackage{esint} % pour les intégrales doubles et triples
\usepackage{esvect} % pour les vecteurs

\usepackage{tikz,tkz-tab} % pour les dessins

\usepackage{pgfplots}
\usepgfplotslibrary{fillbetween}
\usepackage{pgfmath-xfp}

\usetikzlibrary{bending}

\usetikzlibrary{calc,shadows,arrows,shapes,patterns,matrix}
\usetikzlibrary{decorations.pathmorphing}
\usetikzlibrary{fadings}
\usetikzlibrary{external}
\usetikzlibrary{positioning}
\usetikzlibrary{arrows}
\usetikzlibrary{backgrounds}
% pour utiliser \mathcal 

%macros personnelles
\newcommand{\R}{\mathbb{R}}
\newcommand{\N}{\mathbb{N}}
\newcommand{\Q}{\mathbb{Q}}
\newcommand{\Z}{\mathbb{Z}}
\newcommand{\D}{\mathbb{D}}
\newcommand{\C}{\mathbb{C}}

\newcommand{\Rr}{\mathbb{R}}
\newcommand{\Nn}{\mathbb{N}}
\newcommand{\Qq}{\mathbb{Q}}
\newcommand{\Zz}{\mathbb{Z}}
\newcommand{\Dd}{\mathbb{D}}
\newcommand{\Cp}{\mathbb{C}}
\newcommand{\Kk}{\mathbb{K}}

\newcommand{\EX}{\mathbb{E}(X)}
\newcommand{\E}{\mathbb{E}}

\newcommand{\im}{{\mathrm{i}}}\let\ii\im

\DeclareMathOperator{\var}{Var}
\DeclareMathOperator{\cov}{Cov}
\DeclareMathOperator{\prob}{P}
\DeclareMathOperator{\tr}{tr}

\DeclareMathOperator{\Int}{Int}
\DeclareMathOperator{\Ext}{Ext}
\DeclareMathOperator{\Fr}{Fr}

\renewcommand{\Re}{\mathrm{Re}}
\renewcommand{\Im}{\mathrm{Im}}

\newcommand{\dd}{\textup{d}}
\newcommand{\dr}[2]{\dfrac{\partial {#1}}{\partial {#2}}}
\newcommand{\ddr}[2]{\dr{^2 {#1}}{{#2}^2}}
\newcommand{\ddrxy}[3]{\dfrac{\partial ^2{#1}}{\partial {#2}\ \partial {#3}}}
\newcommand{\GF}[1]{{\mathbb F}_{#1}}
\newcommand{\grad}{\overrightarrow{\textup{grad}}\ }
\newcommand{\tend}{\longrightarrow}
\newcommand{\transpose}{{}^t\!}

\newcommand{\pgcd}{\mathop{\mathrm{pgcd}}\nolimits} 

\newcommand{\drawOpenRightHook}[3]{
	\draw[#3, thick] ({#1 + 0.1}, {#2 + 0.02}) -- (#1,#2 + 0.02) -- (#1,#2 - 0.02) -- ({#1 + 0.1}, {#2 - 0.02});
}

\makeatletter
\pgfmathdeclarefunction{erf}{1}{%
  \begingroup
    \pgfmathparse{#1 > 0 ? 1 : -1}%
    \edef\sign{\pgfmathresult}%
    \pgfmathparse{abs(#1)}%
    \edef\x{\pgfmathresult}%
    \pgfmathparse{1/(1+0.3275911*\x)}%
    \edef\t{\pgfmathresult}%
    \pgfmathparse{%
      1 - (((((1.061405429*\t -1.453152027)*\t) + 1.421413741)*\t 
      -0.284496736)*\t + 0.254829592)*\t*exp(-(\x*\x))}%
    \edef\y{\pgfmathresult}%
    \pgfmathparse{(\sign)*\y}%
    \pgfmath@smuggleone\pgfmathresult%
  \endgroup
}
\makeatother

\pgfmathdeclarefunction{gauss}{3}{%
    \pgfmathparse{1/(#3*sqrt(2*pi))*exp(-((#1-#2)^2)/(2*#3^2))}%
}

\def\densnorm#1{1/sqrt(2*pi)*exp(-0.5*(#1*#1))}

\newcommand\mynetwork[1]{
	\def\numpart{\the\numexpr\value{part}}
	\begin{center} 
		\begin{tikzpicture}
			\pgfmathsetmacro\k{1+2*\numpart}  
			\pgfmathsetmacro\kk{\k +2}
			\pgfmathsetmacro\halfk{\k/2}
			\pgfmathsetmacro\halfkk{\kk/2}
			\pgfmathsetmacro\scale{3/\k}
			\def\layersep{4 cm}
			\tikzstyle{every pin edge}=[<-,shorten <=1pt,thick]
			\tikzstyle{neuron}=[circle,fill=black!25,minimum size=10*\scale pt,inner sep=0pt]
			\tikzstyle{entree}=[];
			\tikzstyle{input neuron}=[neuron, fill=green!50];
			\tikzstyle{output neuron}=[neuron, fill=red!50];
			\tikzstyle{hidden neuron}=[neuron, fill=blue!50];
			\tikzstyle{annot} = [text width=4em, text centered]
			% Premiere couche
			\foreach \name / \y in {1,...,\k}{
				% This is the same as writing \foreach \name / \y in {1/1,2/2,3/3,4/4}
				\path[yshift=-\halfk*\scale cm,yshift=-0.5*\scale cm]
				node[input neuron] (I-\name) at (0,\y*\scale) {};
			}
			%Seconde couche
			\foreach \name / \y in {1,...,\kk}{
				\path[yshift=-\halfkk*\scale cm,yshift=-0.5*\scale cm]
				node[hidden neuron] (H-\name) at (\layersep,\y*\scale cm) {};
			}
			% Connect every node in the input layer with every node in the
			% hidden layer.
			\foreach \source in {1,...,\k}{
				\foreach \dest in {1,...,\kk}{
					\path[thick] (I-\source) edge (H-\dest);
				}
			}        
		\end{tikzpicture}  
	\end{center}
}  

\usepackage{varwidth}
\tikzset{
	block/.style = {
		minimum height=1em,
		inner xsep=.8em, inner ysep=.45em,
		draw=black, thick, %rounded corners,
		execute at begin node={\begin{varwidth}{\linewidth}},
			execute at end node={\end{varwidth}}
	},
	inline/.style = {
		inner sep=.35em, draw=black, thick,
	}
}