% package pour les mathématiques
\usepackage{amsmath,amssymb,amsfonts,amsthm}
\usepackage{mathrsfs}
\usepackage{stmaryrd} % pour les intervalles d'entiers
\usepackage{esint} % pour les intégrales doubles et triples
\usepackage{esvect} % pour les vecteurs

\usepackage{tikz,tkz-tab} % pour les dessins
\usepackage{pgfplots}
\usepgfplotslibrary{fillbetween}
\usepackage{pgfmath-xfp}
\usetikzlibrary{patterns}
\usetikzlibrary{bending}

% pour utiliser \mathcal 

%macros personnelles
\newcommand{\R}{\mathbb{R}}
\newcommand{\N}{\mathbb{N}}
\newcommand{\Q}{\mathbb{Q}}
\newcommand{\Z}{\mathbb{Z}}
\newcommand{\D}{\mathbb{D}}
\newcommand{\C}{\mathbb{C}}

\newcommand{\Rr}{\mathbb{R}}
\newcommand{\Nn}{\mathbb{N}}
\newcommand{\Qq}{\mathbb{Q}}
\newcommand{\Zz}{\mathbb{Z}}
\newcommand{\Dd}{\mathbb{D}}
\newcommand{\Cp}{\mathbb{C}}
\newcommand{\Kk}{\mathbb{K}}

\newcommand{\EX}{\mathbb{E}(X)}
\newcommand{\E}{\mathbb{E}}

\newcommand{\im}{{\mathrm{i}}}\let\ii\im

\DeclareMathOperator{\var}{Var}
\DeclareMathOperator{\cov}{Cov}
\DeclareMathOperator{\prob}{P}
\DeclareMathOperator{\tr}{tr}

\renewcommand{\Re}{\mathrm{Re}}
\renewcommand{\Im}{\mathrm{Im}}

\newcommand{\dd}{\textup{d}}
\newcommand{\dr}[2]{\dfrac{\partial {#1}}{\partial {#2}}}
\newcommand{\ddr}[2]{\dr{^2 {#1}}{{#2}^2}}
\newcommand{\ddrxy}[3]{\dfrac{\partial ^2{#1}}{\partial {#2}\ \partial {#3}}}
\newcommand{\GF}[1]{{\mathbb F}_{#1}}
\newcommand{\grad}{\overrightarrow{\textup{grad}}\ }
\newcommand{\tend}{\longrightarrow}
\newcommand{\transpose}{{}^t\!}

\newcommand{\pgcd}{\mathop{\mathrm{pgcd}}\nolimits} 

\newcommand{\drawOpenRightHook}[3]{
	\draw[#3, thick] ({#1 + 0.1}, {#2 + 0.02}) -- (#1,#2 + 0.02) -- (#1,#2 - 0.02) -- ({#1 + 0.1}, {#2 - 0.02});
}

\makeatletter
\pgfmathdeclarefunction{erf}{1}{%
  \begingroup
    \pgfmathparse{#1 > 0 ? 1 : -1}%
    \edef\sign{\pgfmathresult}%
    \pgfmathparse{abs(#1)}%
    \edef\x{\pgfmathresult}%
    \pgfmathparse{1/(1+0.3275911*\x)}%
    \edef\t{\pgfmathresult}%
    \pgfmathparse{%
      1 - (((((1.061405429*\t -1.453152027)*\t) + 1.421413741)*\t 
      -0.284496736)*\t + 0.254829592)*\t*exp(-(\x*\x))}%
    \edef\y{\pgfmathresult}%
    \pgfmathparse{(\sign)*\y}%
    \pgfmath@smuggleone\pgfmathresult%
  \endgroup
}
\makeatother

\pgfmathdeclarefunction{gauss}{3}{%
    \pgfmathparse{1/(#3*sqrt(2*pi))*exp(-((#1-#2)^2)/(2*#3^2))}%
}

\def\densnorm#1{1/sqrt(2*pi)*exp(-0.5*(#1*#1))}