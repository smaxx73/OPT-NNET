\section{Perceptron}

%--------------------------------------------------------------------
\subsection{Perceptron linéaire}

\index{perceptron!lineaire@linéaire}

Le principe du perceptron linéaire est de prendre des valeurs en entrées, de faire un calcul simple et de renvoyer une valeur en sortie. Les calculs dépendent de paramètres propres à chaque perceptron.

\myfigure{1}{
	\tikzinput{fig_neurones_01}
}



Le calcul effectué par un perceptron se décompose en deux phases : un calcul par une fonction linéaire $f$, suivi d'une fonction d'activation $\phi$.

\myfigure{1}{
	\tikzinput{fig_neurones_02}
}


Détaillons chaque phase.
\begin{itemize}
	\item \textbf{Partie linéaire.} Le perceptron est d'abord muni de  \trouer{poids}\index{poids} $a_1,\ldots,a_n$
	qui déterminent une fonction linéaire 
	$$f(x_1,\ldots,x_n) = a_1 x_1 + a_2 x_2 + \cdots + a_n x_n.$$
	\item \textbf{Fonction d'activation.}\index{fonction d'activation} La valeur renvoyée par la fonction linéaire $f$ est ensuite composée par une fonction d'activation $\phi$.
	
	\item \textbf{Sortie.} La valeur de sortie est donc $\phi(a_1 x_1 + a_2 x_2 + \cdots + a_n x_n)$.
\end{itemize}

Dans ce chapitre, la fonction d'activation sera (presque) toujours la fonction marche de Heaviside :
$$\begin{cases}
	H(x) = 1 & \text{ si } x \ge 0, \\
	H(x) = 0  & \text{ si } x < 0. \\
\end{cases}$$

\myfigure{1}{
	\tikzinput{fig_neurones_04}
}



Voici ce que fait un perceptron linéaire de poids $a_1,\ldots,a_n$ et de fonction d'activation la fonction marche de Heaviside :

\myfigure{0.8}{
	\tikzinput{fig_neurones_03}
}


On peut donc définir ce qu'est un perceptron. Un   \trouer{perceptron linéaire} à $n$ variables et de fonction d'activation la fonction marche de Heaviside est la donnée de $n$ coefficients réels $a_1,\ldots,a_n$ auxquels est associée la fonction $F : \Rr \to \Rr$ définie par $F = H \circ f$, c'est-à-dire :
$$\begin{cases}
	F(x_1,\ldots,x_n) = 1 & \text{ si } a_1 x_1 + a_2 x_2 + \cdots + a_n x_n \ge 0, \\
	F(x_1,\ldots,x_n) = 0  & \text{ sinon.} \\
\end{cases}$$



\begin{exemple}{}{}
	
	Voici un perceptron à deux entrées. Il est défini par les poids $a=2$ et $b=3$.
	
	\myfigure{0.8}{
		\tikzinput{fig_neurones_05}
	}
	
	
	\begin{itemize}
		\item \textbf{Formule.}
		
		Notons $x$ et $y$ les deux réels en entrée.
		La fonction linéaire $f$ est donc 
		$$f(x,y) = 2x+3y.$$
		
		La valeur en sortie est donc :
		$$\begin{cases}
			F(x,y) = 1 & \text{ si } 2x+3y \ge 0 \\
			F(x,y) = 0  & \text{ sinon.} \\
		\end{cases}$$
		
		\myfigure{0.8}{
			\tikzinput{fig_neurones_06}
		}
		
		\item \textbf{\'Evaluation.} 
		Utilisons ce perceptron comme une fonction. Que renvoie le perceptron pour la valeur d'entrée $(x,y) = (4,-1)$ ?
		On calcule $f(x,y) = 2x+3y = 5$. Comme $f(x,y)\ge0$, alors la valeur de sortie est donc $F(x,y) = 1$.
		
		Recommençons avec $(x,y) = (-3,1)$. Cette fois $f(x,y) = -3 < 0$ donc $F(x,y) = 0$.
		
		L'entrée $(x,y) = (6,-4)$ est \og{}à la limite\fg{} car $f(x,y)=0$ ($0$ est l'abscisse critique pour la fonction marche de Heaviside).
		On a $F(x,y) = 1$.
		
		\item \textbf{Valeurs de la fonction.}
		
		La fonction $F$ prend seulement deux valeurs : $0$ ou $1$. La frontière correspond aux points $(x,y)$ tels que
		$f(x,y)=0$, c'est-à-dire à la droite $2x+3y=0$.  
		Pour les points au-dessus de la droite (ou sur la droite) la fonction $F$ prend la valeur $1$ ;
		pour les points en-dessous de la droite, la fonction $F$ vaut $0$.
		
		\myfigure{0.8}{
			\tikzinput{fig_neurones_07}
		}  
		
	\end{itemize}
\end{exemple}

\textbf{Notation.}

Nous représentons un perceptron par une forme plus condensée : 
sous la forme d'un   \trouer{neurone}, avec des poids sur les arêtes d'entrées. 
Nous précisons en indice la fonction d'activation utilisée  $\phi$.
Si le contexte est clair cette mention est omise.

\myfigure{0.7}{
	\tikzinput{fig_neurones_08}
}  

Voici le neurone à deux variables de l'exemple précédent.
\myfigure{0.7}{
	\tikzinput{fig_neurones_09}
}  

\begin{exemple}{}{}
	Voici deux catégories de points : des ronds bleus et des carrés rouges. Comment trouver un perceptron qui les sépare ?
	
	\myfigure{0.7}{
		\tikzinput{fig_neurones_10}
	} 
	
	Il s'agit donc de trouver les deux poids $a$ et $b$ d'un perceptron, dont la fonction associée
	$F$ vérifie $F(x,y)=1$ pour les coordonnées des carrés et $F(x,y)=0$ pour les ronds.
	
	\myfigure{0.7}{
		\tikzinput{fig_neurones_11}
	} 
	
	Trouvons une droite qui les sépare. Par exemple, la droite d'équation $4x-y=0$ sépare les ronds des carrés. On définit donc le neurone avec les poids $a=4$ et $b=-1$.
	Si $(x,y)$ sont les coordonnées d'un carré alors on a bien $F(x,y)=1$ et pour un rond $F(x,y)=0$.
	
	\begin{center}
		\begin{minipage}{0.35\textwidth}
			\myfigure{0.5}{
				\tikzinput{fig_neurones_12}
			}
		\end{minipage}
		\begin{minipage}{0.45\textwidth}
			\myfigure{0.6}{
				\tikzinput{fig_neurones_13}
			}
		\end{minipage}
	\end{center}
	
	
\end{exemple}