\renewcommand{\thesection}{\Roman{section}} % I, II, III, IV, ...
\renewcommand{\thesubsection}{\arabic{subsection}} % 1, 2, 3, 4, ...

\usepackage[Sonny]{fncychap} %style de chapitre

\usepackage{xcolor}
\definecolor{vert}{RGB}{80,194,66}
\definecolor{bleu}{RGB}{0,136,206}
\definecolor{rouge}{RGB}{255,56,36}
\definecolor{orange}{RGB}{255,153,0}
\definecolor{violet}{RGB}{163,73,164}
\definecolor{jaune}{RGB}{255,204,0}
\definecolor{gris}{RGB}{128,128,128}
\definecolor{eggplant}{cmyk}{0.04, 0.24, 0, 0.47}
%rouge très foncé :
\definecolor{rougefonce}{RGB}{139,0,0}



\usepackage[explicit,nobottomtitles*]{titlesec}

\titleformat{\section}
{\normalfont\Large\bfseries\sffamily}
{\llap{\colorbox{rougefonce}{\makebox[3em][r]{\textcolor{white}
				{\thesection -}}}\hspace{1em}}}{0pt}{#1}

\titleformat{\subsection}
{\color{rougefonce}\large\bfseries\sffamily}
{\thesubsection.}{1em}{#1}

\definecolor{titlecolour}{RGB}{139,0,0} % Remplacez 0,0,0 par les valeurs RGB de votre choix


\titleformat{\chapter}[display]
{\bfseries\color{titlecolour}}
{\parbox[c][20mm][c]{40mm}{}
	\hfill
	\rotatebox[origin=c]{90}{\normalfont\color{black}\Large{\textsf{\chaptertitlename}}}
	\hspace{2mm}%
	{\colorbox{titlecolour}{\parbox[c][28mm][c]{28mm}{\centering\color{white}\fontsize{80}{80}\selectfont\thechapter}}}}
{10pt}
{\titlerule[4pt]\vskip3pt\titlerule[2pt]\vskip4pt\filcenter\Huge\sffamily #1}





% environnements théorèmes avec tcolorbox

\usepackage[most]{tcolorbox} % pour les boîtes

\newtcbtheorem[number within=chapter]{theoreme}{Théorème}%
{colback=red!5,colframe=rouge!35!black,fonttitle=\bfseries}{th}

\newtcbtheorem[number within=chapter]{proposition}{Proposition}%
{colback=orange!0,colframe=orange!35!black,fonttitle=\bfseries}{pr}

\newtcbtheorem[number within=section]{corollaire}{Corollaire}%
{colback=orange!0,colframe=orange!35!black,fonttitle=\bfseries}{co}

\newtcbtheorem[number within=section]{lemme}{Lemme}%
{colback=orange!0,colframe=orange!35!black,fonttitle=\bfseries}{le}

\newtcbtheorem[number within=section]{definition}{Définition}%
{boxrule=1pt, arc=1mm, colback=vert!0,colframe=vert!60!black,fonttitle=\bfseries}{de}


\tcbset {
	base/.style={
		breakable,
		arc = 0mm,
		bottomtitle=0.5mm,
		boxrule=0mm,
		colbacktitle=black!10!white,
		coltitle=black,
		fonttitle=\bfseries,
		left=2.5mm,
		leftrule=3mm,
		right=3.5mm,
		title={#1},
		toptitle=0.75mm
	}
}

\definecolor{brandblue}{rgb}{0.34,0.7,1}

\newtcbtheorem[number within=subsection]{exemple}{Exemple}{base={#1}}{ex}

\usepackage{awesomebox}

\newcommand{\evidence}[1]{{\textbf{\emph{#1}}}}
%\newcommand{\trouer}[1]{{\textbf{\emph{#1}}}}

%===================images tikz exo7 =====================================
\usetikzlibrary{calc,shadows,arrows.meta,patterns,matrix}
% --- Pour pré-compiler les images en pdf on utilise 'external'
\usetikzlibrary{external}
\newcommand{\tikzinput}[1]{%
	%\tikzsetnextfilename{\import@path tikzcach/#1\cachsuffix}%
	\input{figures/#1.tikz}%
}
\newcommand{\myfigure}[2]{% entrée : échelle, fichier(s) figure à inclure
	\begin{center}\small
		\tikzstyle{every picture}=[scale=1.1*#1]% mise en échelle + 10% (automatiquement annulé à la fin du groupe)
		#2
\end{center}}
\definecolor{myred}{rgb}{0.93,0.26,0}
\definecolor{myorange}{rgb}{0.73,0.36,0}

\newcommand{\beameronly}[1]{}